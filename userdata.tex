% ===============================
% Uživatelská nastavení (vyplní student/ka)
% ===============================

% --- typ práce + verze ---
% "soc" nebo "seminar"
\worktype{seminar}
% "work" (pracovní, červené upozornění s datumem a časem) nebo "final"       
\versiontype{final}      

% --- obor: zadejte pouze číslo; název se doplní automaticky ---
\fieldnum{14}

% --- názvy práce ---
\titlecz{Plný název práce v češtině}
\titleen{Full English Title of the Work}

% --- škola / kraj / město / rok ---
\school{Gymnázium Kladno, nám. E. Beneše 1573, 272~01 Kladno}
\region{Středočeský kraj}
\city{Kladno}
\yearofwork{2025}

% --- autoři (jméno, město, datum pro prohlášení); přidej více řádků dle potřeby ---
\authorsClear
\addAuthor{Jaroslav Holeček}{Kladně}{29.\,října~2025}
\addAuthor{Jana Novotná}{Praze}{29.\,října~2025}
% \addAuthor{Dorota Máchalová}{Praze}{29.\,října~2025}
% "Autor", "Autorka", "Autoři" dle potřeby
\authorsTitle{Autoři} 

% --- vedoucí a konzultanti ---
% Vedoucí práce
\advisor{Mgr. Petr Vedoucí}
% Konzultanti práce (nepovinné - pokud nejsou, ponech prázdné \consultants{})
\consultants{doc. PhDr. Jana Nováková, Ph.D.}
 % "Konzultant", "Konzultantka", "Konzultanti" nebo cokoliv dle potřeby
\consultantsTitle{Konzultantka}

% --- poděkování ---
\acknowledgements{Děkuji svým konzultantům a rodině za podporu.}

% --- anotace & klíčová slova (CZ/EN) ---
\abstractcz{Krátký český souhrn cílů, metod, výsledků a přínosu práce. Abstrakt slouží pro zájemkyni, která hledá vhodnou literaturu pro svou vlastní práci, k rozhodnutí, zda si má přečíst celý dokument nebo zda se následující dlouhý text věnuje nečemu, co ji nezajímá.}
\keywordscz{klíčové; slovo; šablona; SOČ}

\abstracten{A short English abstract summarizing aims, methods, results and contribution.}
\keywordsen{keywords; template; SOC}

% ===============================
% Doplňkové údaje použité pouze pro SEMINÁRNÍ PRÁCI
% (lze nechat prázdné, pokud jde o SOČ)
% ===============================
% Název předmětu
\subjectname{Informatika}   
% Třída       
\classgrade{4.A}   
% Datum odevzdání                
\submissiondate{30.\,listopadu~2025} 
