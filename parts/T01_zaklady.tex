\section{Princip \LaTeX u}
\LaTeX\ je systém pro přípravu dokumentů, který umožňuje oddělit obsah dokumentu od jeho formátování. Uživatel píše text a strukturu dokumentu pomocí značek a příkazů, zatímco \LaTeX\ se stará o jeho správné naformátování podle předem definovaných pravidel a stylů. Tento přístup umožňuje snadnou tvorbu profesionálně vypadajících dokumentů, zejména těch, které obsahují složité prvky jako jsou matematické rovnice, tabulky nebo obrázky.

\subsection{Šablona}
V této šabloně je přednastaveno základní formátování dokumentu pro seminární práce na Gymnáziu Kladno a pro práce v rámci SOČ.

V souboru \texttt{userdata.tex} se jistě velmi rychle zorientujete. Nastavíte zde, zda pracujete na seminární práci nebo na práci v rámci SOČ a také, zda aktuální verze je verzí pracovní, nebo verzí finální pro odevzdání.
Dále zde nastavíte název práce, jméno autora/autorů, vedoucího práce, rok odevzdání a také anotaci práce v češtině a angličtině. Všechny tyto údaje se následně automaticky doplní na titulní stranu.

V běžném textu využijete nejčastěji následující prvky:

\subsection{Odstavce}

Jednotlivé odstavce se píšou jako samostatné bloky textu oddělené prázdným řádkem. Obvykle dokonce jako pouze jeden řádek, na kterém bude mnoho jednotlivých vět.
Pokud mezi řádky ve vstupním textu nebude žádný prázdný řádek, \LaTeX\ je spojí do jednoho odstavce, jako je tento příklad s předchozí a touto větou (řádky ve vstupním souboru).

Pokud mezi řádky ve vstupním textu necháte volný řádek, \LaTeX\ to interpretuje jako konec odstavce a začne nový odstavec, jako je tento příklad (tento odstavec, který právě čtete). Povšimněte si automatického odsazení prvního řádku nového odstavce.

\subsection{Části práce a základní formátování textu}
V dokumentu můžete pomocí označení vytvářet kapitoly \texttt{\textbackslash chapter\{\}}, podkapitoly \texttt{\textbackslash section\{\}}, podpodkapitoly \texttt{\textbackslash subsection\{\}}. Tyto příkazy automaticky číslují části dokumentu a vytvářejí obsah (pokud vložíte příkaz pro jeho generování - podrobnosti \ref{sec:generovane_obsahy}).

Podobně lze také formátovat text pomocí příkazů pro \textbf{tučný text} \texttt{\textbackslash textbf\{\}}, \textit{kurzívu} \texttt{\textbackslash textit\{\}}, \underline{podtržený text} \texttt{\textbackslash underline\{\}} a \texttt{monospace (strojopisný) text} \texttt{\textbackslash texttt\{\}}. Jistě se vám budou také hodit uvozovky \uv{pro citace} pomocí příkazu \texttt{\textbackslash uv\{\}}. 

Trochu jinak lze formátovat text také pomocí prostředí \texttt{verbatim}, které zachovává všechny mezery a zalomení řádků přesně tak, jak jsou ve zdrojovém textu napsány. Takové prostředí začíná značkou \texttt{\textbackslash begin\{verbatim\}} a končí značkou \texttt{\textbackslash end\{verbatim\}}. Cokoliv mezi těmito dvěma příkazy bude zobrazeno přesně tak, jak je napsáno, včetně mezer a nových řádků.
\begin{verbatim}
To je užitečné
  pro ukázky kódu

    nebo jiný text,
      kde je důležité
  
        přesné formátování.
\end{verbatim}

Podobně lze vytvářet seznamy i seznamy v seznamech a seznamy v seznamech v seznamech.
\begin{itemize}
  \item Lze vytvářet
  \item nečíslované seznamy
  \begin{itemize}
    \item které automaticky mění odrážky pro vnořené seznamy
      \item jak se seznamy dělají se podívejte do zdrojového souboru
      \item nebo do tutoriálu k \LaTeX u na \href{https://www.overleaf.com/learn/latex/Learn_LaTeX_in_30_minutes}{Overleaf}
  \end{itemize}
  \item také lze vytvářet
  \item číslované seznamy
  \begin{enumerate}
    \item které automaticky číslují položky
    \item a také
    \begin{enumerate}
      \item mění číslování
      \item pro vnořené seznamy
    \end{enumerate}
    \item ale moc to se seznamy nepřehánějte
    \item obvykle jsou méně přehledné
    \item než si myslíte
  \end{enumerate}
\end{itemize}

\paragraph{Odstavec}
Pojemnovat si dokonce můžete i odstavec :-)

