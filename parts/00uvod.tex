Stručně uveďte cíl práce, kontext, ve kterém práci vypracováváte (jaký je např. aktuální stav v dané oblasti), přehled použitých metod, stručně shrnutý výsledek, kterého se vám podařilo dosáhnout a strukturu práce.

Tato šablona slouží jako výchozí bod pro tvorbu seminární práce na Gymnáziu Kladno nebo práce v rámci SOČ. Text obsahuje vysvětlení a ukázku, jak v práci používat různé prvky jako jsou obrázky, tabulky, rovnice, kód nebo citace. Dále také obsahuje vysvětlení, jak nastavit uživatelská data jako je název práce, autoři, vedoucí, anotace a další. Vznikla za účelem ukázky žákyním a žákům, jak se v praxi řeší, pokud chce společnost nebo škola mít sjednocený formát dokumentů a zároveň nedává smysl zatěžovat každého uživatele zvlášť nastavováním všech detailů formátování. Takové formátování je totiž z principu pro všechny práce stejné (pokud mají být výsledné práce stejné) a nechat jej nastavovat každého uživatele zvlášť je mrhání časem a vede k nekonzistenci.
