V odborných pracích se často používají tabulky a obrázky k lepšímu znázornění dat a informací. Pokud je vložíme správně, vytvoří nám \LaTeX na základě těchto prvků i jejich automatické seznamy.

\section{Tabulky}
Tabulky se vkládají pomocí prostředí \texttt{table} a \texttt{tabular}. Prostředí \texttt{table} slouží k umístění tabulky do dokumentu a umožňuje přidat popisek a označení pro referencování. Prostředí \texttt{tabular} pak definuje samotnou strukturu tabulky, včetně počtu sloupců a jejich zarovnání.

Pro vytvoření tabylky doporučuji nástroj jako je \href{https://www.tablesgenerator.com/}{Tables Generator}\footnote{Dostupné online na https://www.tablesgenerator.com/}, který umožňuje snadno vytvářet tabulky a generovat odpovídající \LaTeX kód.

\begin{table}[h!]
  \centering
  \begin{tabular}{|c|c|c|}
    \hline
    Sloupec 1 & Sloupec 2 & Sloupec 3 \\
    \hline
    Data 1 & Data 2 & Data 3 \\
    Data 4 & Data 5 & Data 6 \\
    \hline
  \end{tabular}
  \caption{Ukázková tabulka}
  \label{tab:ukazkova_tabulka}
\end{table}
Tabulku můžeme v textu odkázat pomocí příkazu \verb|\ref{tab:ukazkova_tabulka}|, který vytvoří odkaz na číslo tabulky: \ref{tab:ukazkova_tabulka}.

\section{Obrázky}
Obrázky se vkládají pomocí prostředí \texttt{figure}. Toto prostředí umožňuje přidat obrázek do dokumentu, přidat popisek a označení pro referencování. Obrázky se obvykle vkládají pomocí příkazu \verb|\includegraphics|, který načte obrázek z externího souboru - cesta k tomuto souboru se zapisuje vzhledem k souboru, ve kterém obrázek používáte. Pokud chcete \uv{o adresář výše}, zapisujete \texttt{../}. Můžete také nastavit velikost obrázku pomocí parametrů jako \texttt{width} nebo \texttt{height} - obvykle vzhledem k šířce textu \texttt{0.6\textbackslash textwidth}.

\begin{figure}[h!]
  \centering
  \includegraphics[width=0.6\textwidth]{img/logo_skoly.png}
  \caption{Ukázkový popisek obrázku, v popisku by měl být uveden zdroj, odkud jste obrázek převzali. Zdroj obrázku: webové stránky Gymnázia Kladno \cite{GymnasiumKladno}}
  \label{fig:ukazkovy_obrazek}
\end{figure}

\begin{figure}[h!]
    \centering
    \begin{minipage}{0.45\textwidth}
        \centering
        \includegraphics[width=\textwidth]{img/logo_skoly.png}
        \caption{Vedle}
        \label{fig:obrazek1}
    \end{minipage}
    \hfill
    \begin{minipage}{0.45\textwidth}
        \centering
        \includegraphics[width=\textwidth]{img/logo_skoly.png}
        \caption{sebe}
        \label{fig:obrazek2}
    \end{minipage}
\end{figure}

\begin{figure}[h!]
    \centering
    \begin{minipage}{0.45\textwidth}
        \centering
        \includegraphics[width=\textwidth]{img/logo_skoly.png}
    \end{minipage}
    \hfill
    \begin{minipage}{0.45\textwidth}
        \centering
        \includegraphics[width=\textwidth]{img/logo_skoly.png}
    \end{minipage}
    \caption{Popisek pro oba obrázky zároveň}
\end{figure}

Obrázek můžeme v textu odkázat pomocí příkazu \verb|\ref{fig:ukazkovy_obrazek}|, který vytvoří odkaz na číslo obrázku: \ref{fig:ukazkovy_obrazek}.


\section{Rovnice}
Rovnice a matematické vztahy lze vkládat několika způsoby.

Lze ji vložit přímo do textu mezi dva znaky dolaru, například $E=mc^2$. 

Pro samostatně číslované rovnice se používá prostředí \texttt{equation}:
\begin{equation} \label{eq:sine}
\frac{a}{\sin \alpha} = \frac{b}{\sin \beta} = \frac{c}{\sin \gamma} 
\end{equation}
\begin{equation} \label{eq:cosine}
c^2 = a^2 + b^2 - 2ab \cos \gamma
\end{equation}
Rovnici můžeme v textu odkázat pomocí příkazu \verb|\ref{eq:sine}|, který vytvoří odkaz na číslo rovnice: \ref{eq:sine}.

Pokud rovnici nechceme číslovat, můžeme použít prostředí \texttt{equation*}:
\begin{equation*}
\sum_{n=1}^{\infty} 2^{-n} = 1
\end{equation*}

Pro více řádků rovnic se používá prostředí \texttt{align}:
\begin{align}
f(x) &= x^2 + 2x + 1 \\
     &= (x + 1)^2 \\
     &= x^2 + 2x + 1
\end{align}

Pro další informace neváhejte využít například \href{https://www.overleaf.com/learn/latex/mathematical_expressions}{tutoriál na Overleaf} nebo jiný zdroj o psaní matematiky v \LaTeX u. V současné době je také mocným pomocníkem LLM, které umí generovat \LaTeX kód pro matematické výrazy na základě textového popisu.

\section{Ukázka kódu}
Možná se vám stane, že budete chtít vložit ukázku programovacího jazyka. Ktomu slouží prostředí \texttt{lstlisting}.

\begin{lstlisting}[language=Python, caption={Krátký Python kód}]
for i in range(5):
    print(i)
\end{lstlisting}

