% určuje sablonu dokumentu
\documentclass{sablona/socthesis}

% načte data, která jste zapsali do souboru userdata.tex
% ===============================
% Uživatelská nastavení (vyplní student/ka)
% ===============================

% --- typ práce + verze ---
% "soc" nebo "seminar"
\worktype{seminar}
% "work" (pracovní, červené upozornění s datumem a časem) nebo "final"       
\versiontype{final}      

% --- obor: zadejte pouze číslo; název se doplní automaticky ---
\fieldnum{14}

% --- názvy práce ---
\titlecz{Plný název práce v češtině}
\titleen{Full English Title of the Work}

% --- škola / kraj / město / rok ---
\school{Gymnázium Kladno, nám. E. Beneše 1573, 272~01 Kladno}
\region{Středočeský kraj}
\city{Kladno}
\yearofwork{2025}

% --- autoři (jméno, město, datum pro prohlášení); přidej více řádků dle potřeby ---
\authorsClear
\addAuthor{Jaroslav Holeček}{Kladně}{29.\,října~2025}
\addAuthor{Jana Novotná}{Praze}{29.\,října~2025}
% \addAuthor{Dorota Máchalová}{Praze}{29.\,října~2025}
% "Autor", "Autorka", "Autoři" dle potřeby
\authorsTitle{Autoři} 

% --- vedoucí a konzultanti ---
% Vedoucí práce
\advisor{Mgr. Petr Vedoucí}
% Konzultanti práce (nepovinné - pokud nejsou, ponech prázdné \consultants{})
\consultants{doc. PhDr. Jana Nováková, Ph.D.}
 % "Konzultant", "Konzultantka", "Konzultanti" nebo cokoliv dle potřeby
\consultantsTitle{Konzultantka}

% --- poděkování ---
\acknowledgements{Děkuji svým konzultantům a rodině za podporu.}

% --- anotace & klíčová slova (CZ/EN) ---
\abstractcz{Krátký český souhrn cílů, metod, výsledků a přínosu práce. Abstrakt slouží pro zájemkyni, která hledá vhodnou literaturu pro svou vlastní práci, k rozhodnutí, zda si má přečíst celý dokument nebo zda se následující dlouhý text věnuje nečemu, co ji nezajímá.}
\keywordscz{klíčové; slovo; šablona; SOČ}

\abstracten{A short English abstract summarizing aims, methods, results and contribution.}
\keywordsen{keywords; template; SOC}

% ===============================
% Doplňkové údaje použité pouze pro SEMINÁRNÍ PRÁCI
% (lze nechat prázdné, pokud jde o SOČ)
% ===============================
% Název předmětu
\subjectname{Informatika}   
% Třída       
\classgrade{4.A}   
% Datum odevzdání                
\submissiondate{30.\,listopadu~2025} 


% nastaví soubor se seznamem literatury ve formátu .bib
\addbibresource{literatura.bib}


% Začátek dokumentu - vše potom bude mezi \begin{document} a \end{document} a zobrazí se v konečném PDF
\begin{document}

% Titulní strana(y)
\maketitle

% Prohlášení
\MakeDeclaration

% Poděkování
\ThanksPage

% Anotace & klíčová slova
\AbstractsPage

% Generovaný obsah
\tableofcontents

\pagestyle{plain} % Odsud se již číslují všechny stránky

% --- Začátek práce ---
\chapter{Úvod}
Stručně uveďte cíl práce, kontext, ve kterém práci vypracováváte (jaký je např. aktuální stav v dané oblasti), přehled použitých metod, stručně shrnutý výsledek, kterého se vám podařilo dosáhnout a strukturu práce.

Tato šablona slouží jako výchozí bod pro tvorbu seminární práce na Gymnáziu Kladno nebo práce v rámci SOČ. Text obsahuje vysvětlení a ukázku, jak v práci používat různé prvky jako jsou obrázky, tabulky, rovnice, kód nebo citace. Dále také obsahuje vysvětlení, jak nastavit uživatelská data jako je název práce, autoři, vedoucí, anotace a další. Vznikla za účelem ukázky žákyním a žákům, jak se v praxi řeší, pokud chce společnost nebo škola mít sjednocený formát dokumentů a zároveň nedává smysl zatěžovat každého uživatele zvlášť nastavováním všech detailů formátování. Takové formátování je totiž z principu pro všechny práce stejné (pokud mají být výsledné práce stejné) a nechat jej nastavovat každého uživatele zvlášť je mrhání časem a vede k nekonzistenci.


\chapter{Teoretická část}
\section{Princip \LaTeX u}
\LaTeX\ je systém pro přípravu dokumentů, který umožňuje oddělit obsah dokumentu od jeho formátování. Uživatel píše text a strukturu dokumentu pomocí značek a příkazů, zatímco \LaTeX\ se stará o jeho správné naformátování podle předem definovaných pravidel a stylů. Tento přístup umožňuje snadnou tvorbu profesionálně vypadajících dokumentů, zejména těch, které obsahují složité prvky jako jsou matematické rovnice, tabulky nebo obrázky.

\subsection{Šablona}
V této šabloně je přednastaveno základní formátování dokumentu pro seminární práce na Gymnáziu Kladno a pro práce v rámci SOČ.

V souboru \texttt{userdata.tex} se jistě velmi rychle zorientujete. Nastavíte zde, zda pracujete na seminární práci nebo na práci v rámci SOČ a také, zda aktuální verze je verzí pracovní, nebo verzí finální pro odevzdání.
Dále zde nastavíte název práce, jméno autora/autorů, vedoucího práce, rok odevzdání a také anotaci práce v češtině a angličtině. Všechny tyto údaje se následně automaticky doplní na titulní stranu.

V běžném textu využijete nejčastěji následující prvky:

\subsection{Odstavce}

Jednotlivé odstavce se píšou jako samostatné bloky textu oddělené prázdným řádkem. Obvykle dokonce jako pouze jeden řádek, na kterém bude mnoho jednotlivých vět.
Pokud mezi řádky ve vstupním textu nebude žádný prázdný řádek, \LaTeX\ je spojí do jednoho odstavce, jako je tento příklad s předchozí a touto větou (řádky ve vstupním souboru).

Pokud mezi řádky ve vstupním textu necháte volný řádek, \LaTeX\ to interpretuje jako konec odstavce a začne nový odstavec, jako je tento příklad (tento odstavec, který právě čtete). Povšimněte si automatického odsazení prvního řádku nového odstavce.

\subsection{Části práce a základní formátování textu}
V dokumentu můžete pomocí označení vytvářet kapitoly \texttt{\textbackslash chapter\{\}}, podkapitoly \texttt{\textbackslash section\{\}}, podpodkapitoly \texttt{\textbackslash subsection\{\}}. Tyto příkazy automaticky číslují části dokumentu a vytvářejí obsah (pokud vložíte příkaz pro jeho generování - podrobnosti \ref{sec:generovane_obsahy}).

Podobně lze také formátovat text pomocí příkazů pro \textbf{tučný text} \texttt{\textbackslash textbf\{\}}, \textit{kurzívu} \texttt{\textbackslash textit\{\}}, \underline{podtržený text} \texttt{\textbackslash underline\{\}} a \texttt{monospace (strojopisný) text} \texttt{\textbackslash texttt\{\}}. Jistě se vám budou také hodit uvozovky \uv{pro citace} pomocí příkazu \texttt{\textbackslash uv\{\}}. 

Trochu jinak lze formátovat text také pomocí prostředí \texttt{verbatim}, které zachovává všechny mezery a zalomení řádků přesně tak, jak jsou ve zdrojovém textu napsány. Takové prostředí začíná značkou \texttt{\textbackslash begin\{verbatim\}} a končí značkou \texttt{\textbackslash end\{verbatim\}}. Cokoliv mezi těmito dvěma příkazy bude zobrazeno přesně tak, jak je napsáno, včetně mezer a nových řádků.
\begin{verbatim}
To je užitečné
  pro ukázky kódu

    nebo jiný text,
      kde je důležité
  
        přesné formátování.
\end{verbatim}

Podobně lze vytvářet seznamy i seznamy v seznamech a seznamy v seznamech v seznamech.
\begin{itemize}
  \item Lze vytvářet
  \item nečíslované seznamy
  \begin{itemize}
    \item které automaticky mění odrážky pro vnořené seznamy
      \item jak se seznamy dělají se podívejte do zdrojového souboru
      \item nebo do tutoriálu k \LaTeX u na \href{https://www.overleaf.com/learn/latex/Learn_LaTeX_in_30_minutes}{Overleaf}
  \end{itemize}
  \item také lze vytvářet
  \item číslované seznamy
  \begin{enumerate}
    \item které automaticky číslují položky
    \item a také
    \begin{enumerate}
      \item mění číslování
      \item pro vnořené seznamy
    \end{enumerate}
    \item ale moc to se seznamy nepřehánějte
    \item obvykle jsou méně přehledné
    \item než si myslíte
  \end{enumerate}
\end{itemize}

\paragraph{Odstavec}
Pojemnovat si dokonce můžete i odstavec :-)


\section{Rozdělení do souborů}
Až bude mít vaše práce více kapitol a podkapitol a celkově bude rozsáhlá, je vhodné ji rozdělit do více samostatných souborů. To usnadní orientaci v textu a jeho úpravy. V této šabloně jsou jednotlivé části práce rozděleny do samostatných souborů v adresáři \texttt{parts/}.

Hlavní soubor \texttt{main.tex} obsahuje pouze základní nastavení dokumentu a příkazy pro zahrnutí jednotlivých částí práce pomocí příkazů \texttt{\textbackslash input\{\}}. Tím lze snadno přidávat a odebírat jednotlivé části práce bez nutnosti pracovat s jedním velkým souborem.

Můžete také velmi snadno měnit pořadí kapitol nebo přidávat nové kapitoly tím, že upravíte pouze hlavní soubor \texttt{main.tex}.

Novou kapitolu přidáte jednoduše tak, že vytvoříte nový soubor v adresáři \texttt{parts/}, například \texttt{03T\_nova\_kapitola.tex}, a poté jej zahrnete do hlavního souboru \texttt{main.tex} pomocí příkazu:
\begin{verbatim}
\input{parts/03T_nova_kapitola}
\end{verbatim}
na místě mezi ostatními kapitolami.

\subsection{Komentáře}
Velmi užitečnou vlastností LaTeXu je možnost vkládat do zdrojového textu komentáře, které nejsou v konečném dokumentu zobrazeny. Komentář začíná znakem procento \% a pokračuje až do konce řádku. Tento znak můžete použít k dočasnému zakomentování částí textu nebo k přidání poznámek pro sebe nebo ostatní, kteří budou s dokumentem pracovat. Například:
\begin{verbatim}
% Možná budu chtít použít tuto verzi věty.
Nebo se mi více líbí tato verze věty.
\end{verbatim}
V tomto příkladu bude v konečném dokumentu zobrazena pouze druhá věta, zatímco první věta bude ignorována jako komentář - pouhým přepsáním znaku procento \% můžete \uv{přehazovat} mezi jednotlivými formulacemi.


\chapter{Praktická část}
V odborných pracích se často používají tabulky a obrázky k lepšímu znázornění dat a informací. Pokud je vložíme správně, vytvoří nám \LaTeX na základě těchto prvků i jejich automatické seznamy.

\section{Tabulky}
Tabulky se vkládají pomocí prostředí \texttt{table} a \texttt{tabular}. Prostředí \texttt{table} slouží k umístění tabulky do dokumentu a umožňuje přidat popisek a označení pro referencování. Prostředí \texttt{tabular} pak definuje samotnou strukturu tabulky, včetně počtu sloupců a jejich zarovnání.

Pro vytvoření tabylky doporučuji nástroj jako je \href{https://www.tablesgenerator.com/}{Tables Generator}\footnote{Dostupné online na https://www.tablesgenerator.com/}, který umožňuje snadno vytvářet tabulky a generovat odpovídající \LaTeX kód.

\begin{table}[h!]
  \centering
  \begin{tabular}{|c|c|c|}
    \hline
    Sloupec 1 & Sloupec 2 & Sloupec 3 \\
    \hline
    Data 1 & Data 2 & Data 3 \\
    Data 4 & Data 5 & Data 6 \\
    \hline
  \end{tabular}
  \caption{Ukázková tabulka}
  \label{tab:ukazkova_tabulka}
\end{table}
Tabulku můžeme v textu odkázat pomocí příkazu \verb|\ref{tab:ukazkova_tabulka}|, který vytvoří odkaz na číslo tabulky: \ref{tab:ukazkova_tabulka}.

\section{Obrázky}
Obrázky se vkládají pomocí prostředí \texttt{figure}. Toto prostředí umožňuje přidat obrázek do dokumentu, přidat popisek a označení pro referencování. Obrázky se obvykle vkládají pomocí příkazu \verb|\includegraphics|, který načte obrázek z externího souboru - cesta k tomuto souboru se zapisuje vzhledem k souboru, ve kterém obrázek používáte. Pokud chcete \uv{o adresář výše}, zapisujete \texttt{../}. Můžete také nastavit velikost obrázku pomocí parametrů jako \texttt{width} nebo \texttt{height} - obvykle vzhledem k šířce textu \texttt{0.6\textbackslash textwidth}.

\begin{figure}[h!]
  \centering
  \includegraphics[width=0.6\textwidth]{img/logo_skoly.png}
  \caption{Ukázkový popisek obrázku, v popisku by měl být uveden zdroj, odkud jste obrázek převzali. Zdroj obrázku: webové stránky Gymnázia Kladno \cite{GymnasiumKladno}}
  \label{fig:ukazkovy_obrazek}
\end{figure}

\begin{figure}[h!]
    \centering
    \begin{minipage}{0.45\textwidth}
        \centering
        \includegraphics[width=\textwidth]{img/logo_skoly.png}
        \caption{Vedle}
        \label{fig:obrazek1}
    \end{minipage}
    \hfill
    \begin{minipage}{0.45\textwidth}
        \centering
        \includegraphics[width=\textwidth]{img/logo_skoly.png}
        \caption{sebe}
        \label{fig:obrazek2}
    \end{minipage}
\end{figure}

\begin{figure}[h!]
    \centering
    \begin{minipage}{0.45\textwidth}
        \centering
        \includegraphics[width=\textwidth]{img/logo_skoly.png}
    \end{minipage}
    \hfill
    \begin{minipage}{0.45\textwidth}
        \centering
        \includegraphics[width=\textwidth]{img/logo_skoly.png}
    \end{minipage}
    \caption{Popisek pro oba obrázky zároveň}
\end{figure}

Obrázek můžeme v textu odkázat pomocí příkazu \verb|\ref{fig:ukazkovy_obrazek}|, který vytvoří odkaz na číslo obrázku: \ref{fig:ukazkovy_obrazek}.


\section{Rovnice}
Rovnice a matematické vztahy lze vkládat několika způsoby.

Lze ji vložit přímo do textu mezi dva znaky dolaru, například $E=mc^2$. 

Pro samostatně číslované rovnice se používá prostředí \texttt{equation}:
\begin{equation} \label{eq:sine}
\frac{a}{\sin \alpha} = \frac{b}{\sin \beta} = \frac{c}{\sin \gamma} 
\end{equation}
\begin{equation} \label{eq:cosine}
c^2 = a^2 + b^2 - 2ab \cos \gamma
\end{equation}
Rovnici můžeme v textu odkázat pomocí příkazu \verb|\ref{eq:sine}|, který vytvoří odkaz na číslo rovnice: \ref{eq:sine}.

Pokud rovnici nechceme číslovat, můžeme použít prostředí \texttt{equation*}:
\begin{equation*}
\sum_{n=1}^{\infty} 2^{-n} = 1
\end{equation*}

Pro více řádků rovnic se používá prostředí \texttt{align}:
\begin{align}
f(x) &= x^2 + 2x + 1 \\
     &= (x + 1)^2 \\
     &= x^2 + 2x + 1
\end{align}

Pro další informace neváhejte využít například \href{https://www.overleaf.com/learn/latex/mathematical_expressions}{tutoriál na Overleaf} nebo jiný zdroj o psaní matematiky v \LaTeX u. V současné době je také mocným pomocníkem LLM, které umí generovat \LaTeX kód pro matematické výrazy na základě textového popisu.

\section{Ukázka kódu}
Možná se vám stane, že budete chtít vložit ukázku programovacího jazyka. Ktomu slouží prostředí \texttt{lstlisting}.

\begin{lstlisting}[language=Python, caption={Krátký Python kód}]
for i in range(5):
    print(i)
\end{lstlisting}


\section{Citace, bibliografie a odkazy}
Pro citace v textu a tvorbu bibliografie se používá balíček \texttt{biblatex}, který je již načten v šabloně. Citace se vkládají pomocí příkazu \verb|\cite{klíč}|, kde \texttt{klíč} je identifikátor záznamu v souboru \texttt{literatura.bib}. Například citace knihy v textu může vypadat takto \cite{Sturma1} a \cite{Malina1}. V textu obvykle citaci začleníme do věty: Dle \cite{Sturma1} je \dots apod.

Do souboru \texttt{literatura.bib} zapíšete záznamy ve formátu BibTeX pro veškerou literaturu a další zdroje, které jste přečetli. V textu poté odkazujete na tyto záznamy pomocí \verb|\cite{klíč}|. Ve výsledném seznamu literatury se objeví pouze ty záznamy, které jste ve vaší praci citovali a automaticky se formátují podle zvoleného stylu. V souboru \texttt{literatura.bib} tedy můžete bez problému mít i záznamy, které nakonec ve vaší práci nepoužijete.

Jaké všechny informace je potřeba vyplnit pro váš typ dokumentu se můžete dočíst v normě ČSN ISO 690 \href{https://www.citace.com/CSN-ISO-690.pdf}{https://www.citace.com/CSN-ISO-690.pdf}.

\subsection{Reference v rámci práce}
V \LaTeX u můžete také vytvářet odkazy na různé části vaší práce, jako jsou kapitoly, obrázky, tabulky nebo rovnice. K tomu slouží příkazy \verb|\label{klíč}| a \verb|\ref{klíč}|. Pomocí \verb|\label{klíč}| označíte místo, na které chcete odkazovat, a pomocí \verb|\ref{klíč}| vytvoříte odkaz na toto místo. Například, pokud chcete odkazovat na obrázek \ref{fig:ukazkovy_obrazek},bude odkaz vypadat takto. V elektronické verzi PDF na tento odkaz můžete kliknout a budete přesměrováni na daný obrázek (respektive na stránku s daným obrázkem).


\section{Poznámky pod čarou}
Místy se hodí využít poznámky pod čarou, které doplňují podrobnosti, které nejsou nutné pro pochopení textu a zbytečně by rušili souvislost textu. 
Poznámka pod čarou se vloží takto\footnote{Toto je ukázková poznámka, která bohužel nevysvětluje nic podrobnějšího :-(}.

\section{Generované obsahy}
\label{sec:generovane_obsahy}
V \LaTeX u lze generovat různé obsahy, jako je obsah kapitol, seznam obrázků nebo tabulek. Tyto obsahy se vytvářejí pomocí příkazů jako
\begin{description}
  \item \verb|\tableofcontents| - generuje obsah kapitol
  \item \verb|\listoffigures| - generuje seznam obrázků
  \item \verb|\listoftables| - generuje seznam tabulek
\end{description}
Tyto příkazy vložíte na místo, kde chcete, aby se obsah objevil, a \LaTeX\ automaticky vytvoří odpovídající seznamy na základě označení a struktur vašeho dokumentu.


\chapter{Závěr}
V dokumentu jsme se věnovali základnímu využití \LaTeX u pro tvorbu odborných prací. \LaTeX\ je velmi rozsáhlý systém s mnoha doplňkovými balíčky. Pokud by vás zajímalo cokoliv dalšího, nebo byste pro vaši práci potřebovali něco, co v tomoto dokumentu není zmíněné, neváhejte pátrat v tutoriálu, využít pomoci LLM a nebo se obrátit na učitele. 

% --- Bibliografie a seznamy po práci ---
\printbibliography[title={Použitá literatura}]
\listoffigures
\listoftables

% --- Přílohy (volitelné) - lze zakomentovat pomocí % (znaku procento) ---
\appendix
\chapter{První příloha}
Přiložené obrázky, zdrojové kódy, tabulky nebo další materiály, které doplňují text práce.

\chapter{Druhá příloha} % Může být v samostatném souboru
Přiložené obrázky, zdrojové kódy, tabulky nebo další materiály, které doplňují text práce.


\end{document}
